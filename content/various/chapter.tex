\section{Various}

\subsection{Intervals}
	\kactlimport{IntervalContainer.h}
	\kactlimport{IntervalCover.h}
	\kactlimport{ConstantIntervals.h}

\subsection{Misc. algorithms}
	\kactlimport{TernarySearch.h}
	\kactlimport{Karatsuba.h}
	\kactlimport{LIS.h}
	\kactlimport{LCS.h}

\subsection{Dynamic programming}
	\kactlimport{DivideAndConquerDP.h}
	\kactlimport{KnuthDP.h}

\subsection{Debugging tricks}
	\begin{itemize}
		\item \texttt{signal(SIGSEGV, [](int) \{ \_Exit(0); \});} converts segfaults into Wrong Answers.
			Similarly one can catch SIGABRT (assertion failures) and SIGFPE (zero divisions).
			\texttt{\_GLIBCXX\_DEBUG} violations generate SIGABRT (or SIGSEGV on gcc 5.4.0 apparently).
		\item \texttt{feenableexcept(29);} kills the program on NaNs (\texttt 1), 0-divs (\texttt 4), infinities (\texttt 8) and denormals (\texttt{16}).
	\end{itemize}

\subsection{Optimization tricks}
	\subsubsection{Bit hacks}
		\begin{itemize}
			\item Useful identity: $\bigoplus_{x=0}^{a - 1} x = \{0, a - 1, 1, a\}[a \, \mathrm{mod} \, 4]$.
			\item \verb@x & -x@ is the least bit in \texttt{x}.
			\item \verb@for (int x = m; x; ) { --x &= m; ... }@ loops over all subset masks of \texttt{m} (except \texttt{m} itself).
			\item \verb@c = x&-x, r = x+c; (((r^x) >> 2)/c) | r@ is the next number after \texttt{x} with the same number of bits set.
			\item \verb@rep(b,0,K) rep(i,0,1<<K) if (i & 1<<b)@\\
				\verb@    D[i] += D[i^(1<<b)];@ computes all sums of subsets.
		\end{itemize}
	\subsubsection{Pragmas}
		\begin{itemize}
			\item \lstinline{#pragma GCC optimize ("Ofast")} will make GCC auto-vectorize for loops and optimizes floating points better (assumes associativity and turns off denormals).
			\item \lstinline{#pragma GCC target ("avx,avx2")} can double performance of vectorized code, but causes crashes on old machines.
			\item \lstinline{#pragma GCC optimize ("trapv")} kills the program on integer overflows (but is really slow).
		\end{itemize}
	\kactlimport{BumpAllocator.h}
	\kactlimport{SmallPtr.h}
	\kactlimport{BumpAllocatorSTL.h}
	\kactlimport{Unrolling.h}
	\kactlimport{SIMD.h}
